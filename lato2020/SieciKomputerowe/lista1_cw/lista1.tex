\documentclass[a4paper]{article}
\usepackage{mathtools}
\usepackage{titlesec}
\usepackage[utf8]{inputenc}
\usepackage[polish]{babel}
\usepackage{textcomp}
\usepackage[T1]{fontenc}

\title{Lista 1}
\author{Dawid Żywczak}
\date{24 marca 2020}


\begin{document}
\maketitle
\subsection*{Zadanie 1}
Mamy dla każdego adresu w formacie CIDR znaleźć adres sieci, adres rozgłoszeniowy oraz inny adres w sieci.

\begin{itemize}
\item 10.1.2.3/8 - Adres sieci: 10.0.0.0, adres rozgłoszeniowy: 10.255.255.255, jakiś adres: 10.1.2.4
\item 156.17.0.0/16 - Adres sieci: 156.17.0.0, adres rozgłoszeniowy:156.17.255.255, jakiś adres: 156.17.10.10
\item 99.99.99.99/27 - Adres sieci: 99.99.99.96, adres rozgłoszeniowy: 99.99.99.127, jakiś adres: 99.99.99.100
\item 156.17.64.4/30 - Adres sieci: 156.17.64.4, adres rozgłoszeniowy: 156.17.64.7, jakiś adres: 156.17.64.5
\item 123.123.123.123 - Sieć z jednym ip, to jedno jest wszystkim
\end{itemize}

\subsection*{Zadanie 2}
Mamy podzielić sieć 10.10.0.0/16 na 5 podsieci tak, aby każdy adres był w jednej z podsieci.


Adres sieci: 00001010.00001010.0.0, maska: 255.255.0.0. Dzielimy sieć na podsieci:\\
\begin{itemize}
\item Adres sieci: 00001010.00001010.0.0, maska: 11111111.11111111.10000000.0 tzn 10.10.0.0/17\\
Zakres 10.10.0.0 \textrightarrow 10.10.127.255
\item Adres sieci: 00001010.00001010.10000000.0, maska: 255.255.11000000.0 tzn 10.10.128.0/18\\
Zakres 10.10.128.0 \textrightarrow 10.10.191.255
\item Adres sieci: 00001010.00001010.11000000.0, maska: 255.255.11100000.0 tzn 10.10.192.0/19
Zakres 10.10.192.0 \textrightarrow 10.10.123.255
\item Adres sieci: 00001010.00001010.11100000.0, maska: 255.255.11110000.0 tzn 10.10.224.0/20
Zakres 10.10.224.0 \textrightarrow 10.10.139.255
\item Adres sieci: 00001010.00001010.11110000.0, maska: 255.255.11110000.0 tzn 10.10.240.0/20
Zakres 10.10.240.0 \textrightarrow 10.10.255.255
\end{itemize}
Co robimy?\\
Bierzemy adres naszej sieci i dzielimy go na dwie podsieci. Jedną z nich zostawiamy, drugą znowu dzielimy na dwie podsieci itd.\\
Straciliśmy 10 adresów na adresy sieci oraz rozgłoszeniowe, ale na początku mieliśmy ich 2, więc ostatecznie straciliśmy 8 adresów.\\
Najmniejsza wielkość sieci tym sposobem to \begin{math}2^{12}\end{math}, ponieważ zawsze dzielimy jedną z naszych podsieci na pół.

\subsection*{Zadanie 3}
Mamy następującą tablicę routingu:
\begin{itemize}
\item 0.0.0.0/0 $\rightarrow$ do routera A
\item 10.0.0.0/23 $\rightarrow$ do routera B
\item 10.0.2.0/24 $\rightarrow$ do routera B
\item 10.0.3.0/24 $\rightarrow$ do routera B
\item 10.0.1.0/24 $\rightarrow$ do routera C
\item 10.0.0.128/25 $\rightarrow$ do routera B
\item 10.0.1.8/29 $\rightarrow$ do routera B
\item 10.0.1.16/29 $\rightarrow$ do routera B
\item 10.0.1.24/29 $\rightarrow$ do routera B
\end{itemize}
Skorzystajmy z faktu, że zawsze wybierana jest najkonkretniejsza reguła dla danego adresu.
\begin{itemize}
\item 0.0.0.0/0 $\rightarrow$ do routera A
\item 10.0.0.0/22 $\rightarrow$ do routera B
\item 10.0.1.0/24 $\rightarrow$ do routera C
\item 10.0.1.0/27 $\rightarrow$ do routera B
\item 10.0.1.0/29 $\rightarrow$ do routera C
\end{itemize}

\clearpage

\subsection*{Zadanie 4}
Mamy wykonać to co wyżej tylko dla innej tablicy routingu

\begin{itemize}
\item 0.0.0.0/0 $\rightarrow$ do routera A
\item 10.0.0.0/8 $\rightarrow$ do routera B
\item 10.3.0.0/24 $\rightarrow$ do routera C
\item 10.3.0.32/27 $\rightarrow$ do routera B
\item 10.3.0.64/27 $\rightarrow$ do routera B
\item 10.3.0.96/27 $\rightarrow$ do routera B
\end{itemize}

Zauważmy, że możemy zrobić dziurę w dość szczegółowej regule kierującej pakiety do routera C. Tym sposobem otrzymujemy:

\begin{itemize}
\item 0.0.0.0/0 $\rightarrow$ do routera A
\item 10.0.0.0/8 $\rightarrow$ do routera B
\item 10.3.0.0/27 $\rightarrow$ do routera C
\item 10.3.0.192/25 $\rightarrow$ do routera C
\end{itemize}

\subsection*{Zadanie 5}
Powinniśmy uporządkować adresy w porządku malejącym względem długości prefixa tzn. adres o najdłuższym prefixie powinien znajdować się jako pierwszy w tablicy routingu. Dlaczego? Z wykładu wiemy, że adres najlepiej pasujący, to ten który jest dopasowany oraz posiada najdłuższy prefiks (jest najbardziej doprecyzowany). Zatem układając adresy zgodnie z powyższą kolejnością, pierwszy dopasowany adres, będzie również najlepiej pasującym, ponieważ zawiera najdłuższy prefix.

\clearpage
\subsection*{Zadanie 6}
\begin{center}
Krok 0\\
\begin{tabular}{|c *{6}{|c} |c|}\hline
 & A & B & C & D & E & F\\
\hline 
do A & - & 1 & & & &\\
\hline 
do B & 1 & - & 1 & & &\\
\hline 
do C & & 1 & - & & 1 & 1\\
\hline 
do D & & & & - & 1 &\\
\hline 
do E & & & 1 & 1 & - & 1\\
\hline
do F & & & 1 & & 1 & -\\
\hline
\end{tabular}

Krok 1\\
\begin{tabular}{|c *{6}{|c} |c|}\hline
 	 & A & B & C 		 & D & E & F\\
\hline 
do A & - & 1 & 2 (Via B) & & &\\
\hline 
do B & 1 & - & 1 		 & & 2 (Via C) & 2 (Via C)\\
\hline 
do C & 2 (Via B) & 1 & - & 2 (Via E) & 1 & 1\\
\hline 
do D & & & 2 (Via E) & - & 1 & 2 (Via E)\\
\hline 
do E & & 2 (Via C) & 1 & 1 & - & 1\\
\hline
do F & & 2 (Via C) & 1 & 2 (Via E) & 1 & -\\
\hline
\end{tabular}

Krok 2\\
\begin{tabular}{|c *{6}{|c} |c|}\hline
 & A & B & C & D & E & F\\
\hline 
do A & - & 1 & 2 (Via B) &  & 3 (Via C) &  3 (Via C)\\
\hline 
do B & 1 & - & 1 & 3 (Via E) & 2 (Via C) & 2 (Via C)\\
\hline 
do C & 2 (Via B) & 1 & - & 2 (Via E) & 1 & 1\\
\hline 
do D & 			 & 3 (Via C) & 2 (Via E) & - & 1 & 2 (Via E)\\
\hline 
do E & 3 (Via B) & 2 (Via C) & 1 & 1 & - & 1\\
\hline
do F & 3 (Via B) & 2 (Via C) & 1 & 2 (Via E) & 1 & -\\
\hline
\end{tabular}

Krok 3\\
\begin{tabular}{|c *{6}{|c} |c|}\hline
 & A & B & C & D & E & F\\
\hline 
do A & - & 1 & 2 (Via B) & 4 (Via E)  & 3 (Via C) &  3 (Via C)\\
\hline 
do B & 1 & - & 1 & 3 (Via E) & 2 (Via C) & 2 (Via C)\\
\hline 
do C & 2 (Via B) & 1 & - & 2 (Via E) & 1 & 1\\
\hline 
do D & 4 (Via B) & 3 (Via C) & 2 (Via E) & - & 1 & 2 (Via E)\\
\hline 
do E & 3 (Via B) & 2 (Via C) & 1 & 1 & - & 1\\
\hline
do F & 3 (Via B) & 2 (Via C) & 1 & 2 (Via E) & 1 & -\\
\hline
\end{tabular}
\end{center}

\clearpage

\subsection*{Zadanie 7}
Dołóżmy połączenie między routerem A oraz D. Jak zmieni się tablica routingu?
\begin{center}
Krok 0\\
\begin{tabular}{|c *{6}{|c} |c|}\hline
 & A & B & C & D & E & F\\
\hline 
do A & - & 1 & 2 (Via B) & 1  & 3 (Via C) &  3 (Via C)\\
\hline 
do B & 1 & - & 1 & 3 (Via E) & 2 (Via C) & 2 (Via C)\\
\hline 
do C & 2 (Via B) & 1 & - & 2 (Via E) & 1 & 1\\
\hline 
do D & 1 & 3 (Via C) & 2 (Via E) & - & 1 & 2 (Via E)\\
\hline 
do E & 3 (Via B) & 2 (Via C) & 1 & 1 & - & 1\\
\hline
do F & 3 (Via B) & 2 (Via C) & 1 & 2 (Via E) & 1 & -\\
\hline
\end{tabular}

Krok 1\\
\begin{tabular}{|c *{6}{|c} |c|}\hline
 & A & B & C & D & E & F\\
\hline 
do A & - & 1 & 2 (Via B) & 1 & 2 (Via D) &  3 (Via C)\\
\hline 
do B & 1 & - & 1 & 2 (Via A) & 2 (Via C) & 2 (Via C)\\
\hline 
do C & 2 (Via B) & 1 & - & 2 (Via E) & 1 & 1\\
\hline 
do D & 1 & 2 (Via A) & 2 (Via E) & - & 1 & 2 (Via E)\\
\hline 
do E & 2 (Via D) & 2 (Via C) & 1 & 1 & - & 1\\
\hline
do F & 3 (Via B) & 2 (Via C) & 1 & 2 (Via E) & 1 & -\\
\hline
\end{tabular}
\end{center}

\subsection*{Zadanie 8}
Rozważmy dwie sytuacje:\\
1. Zostaje uszkodzone połączenie między D i E. D wysyła informację o tym zdarzeniu do B i C, one aktualizują swoją tablicę routingu i wysyłają wiadomość do A. Wszystko działa, odległości do E ustawione są na nieskończoność.
Co się stanie, kiedy to A wyśle najpierw komunikat do B i C
(zakładając, że zaszła awaria łącza D i E)?\\
2. A wyśle do B i C informację, że ma do E ścieżkę długości 3. One mając ustawione niekończone ścieżki do E, pomyślą, że przez A jest szybciej. Zaktualizują więc swoją tablicę routingu i wyślą do sąsiadów: potrafię dojść do E ścieżką długości 4. W kolejnym kroku tą informację otrzyma C, dla którego ścieżka ta jest krótsza od nieskończoności. Ustawi więc u siebie, że ścieżka do E ma długość 5. Teraz A musi zaktualizować swoją tablicę, bo ścieżka (w B i C) wydłużyła się, więc ustawia u siebie, że ścieżka do E ma długość 4 oraz rozsyła ten komunikat do B i C i tak w kółko. Mamy więc cykl (zliczanie do nieskończoności).
\end{document}