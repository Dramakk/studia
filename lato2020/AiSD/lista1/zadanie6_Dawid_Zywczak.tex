\documentclass[a4paper]{scrartcl}
\usepackage{amsmath}
\usepackage{mathtools}
\usepackage[utf8]{inputenc}
\usepackage[polish]{babel}
\usepackage{textcomp}
\usepackage[T1]{fontenc}
\usepackage{amsthm}
\usepackage{amsfonts}
\newtheorem{definition}{Definicja}
\newtheorem{theorem}{Twierdzenie}
\newtheorem{lemma}{Lemat}
\usepackage{hyperref}
\hypersetup{
    colorlinks,
    citecolor=black,
    filecolor=black,
    linkcolor=black,
    urlcolor=black
}
\usepackage{algorithmic}
\title{Algorytmy i struktury danych}
\subtitle{Zadanie 6, lista 1}
\author{Dawid Żywczak}
\date{6 kwietnia 2020}

\begin{document}
\maketitle
Na wejściu dostajemy ciąg $a_1, a_2, ..., a_n$ taki, że dla każdego i $a_i <= a_{i+1}$. Jedyna operacja jaką możemy na tym ciągu wykonywać, to usuwanie elementów $a_i$ oraz $a_j$, gdy $2a_i \leq a_j$. Mamy skonstruować algorytm, obliczający ile conajwyżej par możemy usunąć.\\
Problem ten rozwiążemy algorytmem zachłannym.\\
\begin{algorithmic}
\STATE $i\gets 1$
\STATE $j \gets \lfloor \frac{n}{2} \rfloor +1$
\STATE $k = 0$
\WHILE{$j \leq n$}
        \IF {$2A[i] \leq A[j]$}
        	\STATE $delete\_pair(a_i, a_j)$
        	\STATE $k \gets k+1$
        	\STATE $i \gets i+1$
		\ENDIF
        \STATE $j \gets j+1$
\ENDWHILE
\RETURN $k$
\end{algorithmic}
Co my tak naprawdę robimy? Zachłannie szukamy pary dla elementów prawej części tablicy. Jeżeli się udaje, usuwamy daną parę i zwiększami nasz licznik usuniętych par o 1. Dlaczego to działa?\\
Załóżmy, że mamy rozwiązanie optymalne, nazwijmy je O, które usuwa k par. Chcę pokazać, że jesteśmy w stanie przekształcić rozwiązanie O na rozwiązanie produkowane przez powyższy algorytm. Jak to zrobić?
\begin{lemma}
Niech L i R będą zbiorami, takimi że dla dowolnej usuwanej pary $(a_i, a_j)$, $a_i \in L$, $a_j \in R$ , zbiór L posiada k elementów, oraz indeksy elemntów w zbiorze R są $\leq \lfloor \frac{n}{2} \rfloor$. Chcemy też by zbiory L i R spełniały założenia zadania, tzn każdy element zbioru L jest mniejszy od najmniejszego elementu ze zbioru R.
\end{lemma}
\begin{proof}
Jak to pokazać? Rozważmy przypadki!\\
1. Weźmy dowolne pary punktów, które możemy usunąć $(a_{i}, a_{i1})$ oraz $(a_j, a_{j1}$, takie że $a_{i1} \leq a_j$. Wtedy pary $(a_i, a_j)$ oraz $(a_{i1}, a_{j1})$ również możemy usunąć, a z tego wynika, że możemy je ułożyć w wyżej opisany sposób.\\
2. Weźmy dowolny punkt $a_i$ o indeksie mniejszym od k. Powinnien należeć do zbioru L, ale załóżmy, że tak nie jest. Weźmy najmniejszy punkt $a_j \in L$ o większym indeksie od i. Wtedy można punkt sparowany z nim sparować z naszym $a_i$, a punkt $a_j$ sparować z parą kolejnego punktu i postępować w ten sposób tak długo, aż każdy punkt będzie miał parę.\\
3. A co jeśli w zbiorze R znajdują się jakieś elementy o indeksie mniejszym od m? Możemy je zamienić na elementy o większych indeksach $\geq m$ korzystając z własności naszego ciągu oraz faktu, że $k \leq m$.
\end{proof}
Teraz wystarczy pokazać, że dla każdego elemntu wyżej zdefiniowanego zbioru L, nasz algorytm znajudje parę.\\
\begin{proof}
Baza:
Element $a_1$ zostanie sparowany z elementem ze zbioru R o najmniejszym indeksie, zatem znajdzie parę z najmniejszym elementem R.
Krok:
Weźmy dowolne $i \leq k$ i załóżmy, że znaleźliśmy pary dla wszystkich elemntów $a_1, a_2,..., a_{i-1}$. Znalezione pary, to kolejne najmniejsze elementy ze zbioru R, zatem w szczególności są one $\geq a_m$. Wtedy punkt $a_i$ znajdzie sobie parę z kojenym najmniejszym punktem, po parze punktu $a_i$, który spełni wymaganą nierówność. Taki punkt na pewno istnieje, bo możemy brać też elementy ze zbioru R.
\end{proof}
Jaką mamy złożoność? Maksymalnie n razy sprawdzimy nasz warunek przechodząc całą tablicę, więc jest to O(n).
\end{document}