\documentclass[a4paper]{scrartcl}
\usepackage{amsmath}
\usepackage{mathtools}
\usepackage{titlesec}
\usepackage[utf8]{inputenc}
\usepackage[polish]{babel}
\usepackage{textcomp}
\usepackage[T1]{fontenc}
\usepackage{amsthm}
\usepackage{amsfonts}
\usepackage{hyperref}
\hypersetup{
    colorlinks,
    citecolor=black,
    filecolor=black,
    linkcolor=black,
    urlcolor=black
}

\title{Zamiennik części kolokwium - zadanie 1}
\subtitle{Rachunek prawdopodobieństwa i statystyka}
\author{Karolina Jeziorska, Dawid Żywczak}
\date{19 kwietnia 2020}


\begin{document}
\maketitle
\tableofcontents
\clearpage
\section{Redukcja problemu}
Gęstośc standardowego rozkładu normalnego wraża się wzorem
\begin{center}
$ f(x)=\frac{1}{\sqrt{2\pi}}\exp(-\frac{x^2}{2})$, $x \in \mathbb{R}$
\end{center}
A zatem dystrybuanta zadana jest wzorem
\begin{center}
$ \phi(t)=\int^{t}_{-\infty} \frac{1}{\sqrt{2\pi}}\exp(-\frac{x^2}{2}) dx$
\end{center}
Jak dowiadujemy się z treści zadania, całka ta nie ma niestety przedstawienia za pomocą funkcji elementarnych. Aby policzyć jej wartość zredukujmy problem do policznnia innej, znacznie prostszej całki:
\begin{center}
$ G(t)=\int^{t}_{0}\exp(-\frac{x^2}{2}) dx$
\end{center}
Aby skorzystać z funckji G udowodnijmy najpierw, że
\begin{center}
$\phi(t) = 1 - \phi(-t)$
\end{center}
\begin{proof}
Zacznijmy od prawej strony równości
\begin{center}
$1 - \phi(-t) = 
\int^{\infty}_{-\infty} \frac{1}{\sqrt{2\pi}}\exp(-\frac{x^2}{2})dx - \int^{-t}_{-\infty} \frac{1}{\sqrt{2\pi}}\exp(-\frac{x^2}{2})dx = \int^{\infty}_{-t} \frac{1}{\sqrt{2\pi}}\exp(-\frac{x^2}{2})dx$
\end{center}
Teraz podstawmy $v = -x$, $dv=-dx$
\begin{center}
$-\int^{-\infty}_{t} \frac{1}{\sqrt{2\pi}}\exp(-\frac{(-v)^2}{2})dv =$\\
$\int^{t}_{-\infty} \frac{1}{\sqrt{2\pi}}\exp(-\frac{v^2}{2})dv = \phi(t)$
\end{center}
\end{proof}
Skorzystajmy teraz z funkcji $G(x)$ w celu obliczenia wartości naszej dystrybuanty
\begin{center}
$\phi(t) = 1 - \phi(-t)$\\
$\phi(t) + \phi(-t) = 1$\\
$\int^{t}_{-\infty} \frac{1}{\sqrt{2\pi}}\exp(-\frac{x^2}{2}) + \int^{-t}_{-\infty} \frac{1}{\sqrt{2\pi}}\exp(-\frac{x^2}{2}) = 1$\\
$\int^{-t}_{-\infty} \frac{1}{\sqrt{2\pi}}\exp(-\frac{x^2}{2}) + 
\int^{0}_{-t} \frac{1}{\sqrt{2\pi}}\exp(-\frac{x^2}{2}) + 
\int^{t}_{0} \frac{1}{\sqrt{2\pi}}\exp(-\frac{x^2}{2}) +
\int^{-t}_{-\infty} \frac{1}{\sqrt{2\pi}}\exp(-\frac{x^2}{2}) = 1$
\end{center}
Korzystając z parzystości funkcji gęstości mamy
\begin{center}
$2\cdot \int^{t}_{0} \frac{1}{\sqrt{2\pi}}\exp(-\frac{x^2}{2}) + 
2\cdot \phi(-t) = 1$\\
$2\cdot \int^{t}_{0} \frac{1}{\sqrt{2\pi}}\exp(-\frac{x^2}{2}) + 
2 - 2\cdot \phi(t) = 1$\\
$\frac{1}{\sqrt{2\pi}}\cdot G(x) + \frac{1}{2} = \phi(t)$
\end{center}
Zatem aby otrzymać wartość dystrybuanty dla danego t, wystarczy obliczyć wartość powyższego wyrażenia korzystając np. z metody Romberga.
\end{document}